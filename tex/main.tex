\documentclass{beamer}
\usepackage{graphicx} % Required for inserting images
\usepackage{colortbl}
\usepackage{adjustbox}
\usepackage{mathtools}

\usepackage{listings}
\usepackage{color}
\usepackage{natbib}
\usepackage{xcolor}

\definecolor{myblue}{RGB}{20,105,176}
\definecolor{constants}{RGB}{127,39,84}
\definecolor{character}{RGB}{100,169,57}
\setbeamertemplate{footline}[page number]
\lstdefinestyle{deltaj}{
        belowcaptionskip=1\baselineskip,
        breaklines=true,
        columns=fullflexible,
        frame=single,
        xleftmargin=\parindent,
        language=Python,
        escapeinside={(*}{*)},
        stepnumber=1,
        tabsize=2,
        numberblanklines=false,
        numberstyle=\normalfont\footnotesize,
        basicstyle=\footnotesize\ttfamily,
        keywordstyle=\bfseries\color{green!40!black},
        commentstyle=\itshape\color{purple!40!black},
        identifierstyle=\color{blue},
        stringstyle=\color{orange}, 
        showstringspaces=false
    }
\lstset{style=deltaj}

\beamertemplatenavigationsymbolsempty
\usetheme{Berlin}

\title[Basic Python]{Basic Python Programming}
\subtitle[POSN Computer]{for POSN Computer}
\author[Pana W.]{Pana Wanitchollakit}
\date{2024}

\begin{document}

\frame{\titlepage}

\begin{frame}{Outline}
    \tableofcontents
\end{frame}


\section{Basic}

\subsection[Data types]{Data types}
\begin{frame}{Data types}
    \begin{block}{Data types}
        \lstinputlisting{source/datatypes.py}
    \end{block}
    \begin{block}{Type conversion}
        \lstinputlisting{source/typeconversion.py}
    \end{block}
\end{frame}


\subsection[I/O]{Input and Output}
\begin{frame}{Input and Output}
    A function named \textbf{input} is used for only \textbf{string} input. \\
    The \textbf{input} function will receive input until you press the $<$Enter$>$.
    \begin{block}{Input}
        \lstinputlisting{source/input.py}
        If you want an integer input, you must convert it to an integer, i.e.,
        \lstinputlisting{source/int_input.py}
    \end{block}
\end{frame}

\begin{frame}{Input and Output}
    A \textbf{print} function is used to show output on your commandline. \\
    Note: \textbf{print} will add an extra character, namely a newline character "\textbackslash n" at the end of your output.
    \begin{block}{Output}
        \lstinputlisting{source/print.py}
    \end{block}
\end{frame}

\subsection[Operator]{Operator}
\subsubsection{String operator}
\begin{frame}{String operator}
    \begin{block}{Plus operator (+)}
        We can concatenate the string using \textbf{+} operator
        \lstinputlisting{source/plus_string.py}
    \end{block}
\end{frame}

\subsubsection{Arithmetic Operators}
\begin{frame}{Arithmetic Operators}
    \begin{block}{Arithmetic}
        \lstinputlisting{source/arithmetic.py}
    \end{block}
\end{frame}

\begin{frame}{Arithmetic operator Precedence}
    \centering
    \begin{tabular}{|c|c|c|c|}
        \hline
        Order & Operator & Associativity Type \\
        \hline
        1     & ()       & -                  \\
        \hline
        2     & **       & right to left      \\
        \hline
        3     & *,/,//   & left to right      \\
        \hline
        4     & +, -     & left to right      \\
        \hline
    \end{tabular}
\end{frame}

\begin{frame}{Arithmetic operator Precedence}
    \begin{example}
        \lstinputlisting{source/arith_precedence/e1.py}
        $$
            10 + (3\times4) + (2^3 \times 4) + 5 = 59
        $$
    \end{example}
\end{frame}

\begin{frame}{Arithmetic operator Precedence}
    \begin{example}
        \lstinputlisting{source/arith_precedence/e2.py}
        $$
            4 \times 5^{3^2} \times 7 = 4 \times 5^9 \times 7
        $$
    \end{example}
\end{frame}

\begin{frame}{Arithmetic operator Precedence}
    \begin{example}
        \lstinputlisting{source/arith_precedence/e3.py}
        $$
            \left \lfloor \frac{\left \lfloor \frac{5 \times 3}{4} \right\rfloor \times 10}{7} \right \rfloor
        $$
        Note: \\
        Floor function \href{https://en.wikipedia.org/wiki/Floor_and_ceiling_functions}{\beamergotobutton{Definition}} : $\lfloor x \rfloor$ \\
        Example: $\lfloor -2.5 \rfloor = -3$, $\lfloor 2.5 \rfloor = 2$
    \end{example}
\end{frame}

\end{document}
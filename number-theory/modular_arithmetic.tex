\documentclass[10pt]{beamer}
\usetheme{Ilmenau}
\usecolortheme{default}
\usefonttheme{professionalfonts}
\setbeamertemplate{footline}[page number]
\beamertemplatenavigationsymbolsempty

% Packages
\usepackage{amsmath}
\usepackage{amssymb}
\usepackage{amsthm}
\usepackage{graphicx}
\usepackage{tikz}
\usepackage{xcolor}



\AtBeginSection{
    \frame{\sectionpage}
}

% Title page information
\title{Modular Arithmetic}
\subtitle{Basic Concepts and Applications}
\author[Wanitchollakit]{None W., Pana W., Sean W.}
\date{}

% Custom commands
\newcommand{\Z}{\mathbb{Z}}
\newcommand{\N}{\mathbb{N}}

\begin{document}

% Title slide
\begin{frame}
    \titlepage
\end{frame}

% Table of contents
\begin{frame}
    \frametitle{Outline}
    \tableofcontents
\end{frame}


\section{Basic Definitions}

\begin{frame}
    \frametitle{Congruence Modulo n}
    \begin{definition}
        Let $n$ be a positive integer. Two integers $a$ and $b$ are \textbf{congruent modulo n} if $n$ divides $(a - b)$.

        We write: $a \equiv b \pmod{n}$
    \end{definition}

    \vspace{0.5cm}

    \begin{block}{Equivalent Definitions}
        The following are equivalent:
        \begin{enumerate}
            \item $a$ and $b$ have the same remainder when divided by $n$
            \item $a - b = kn$ for some integer $k$
            \item $n \mid (a - b)$
            \item $a \equiv b \pmod{n}$
        \end{enumerate}
    \end{block}
\end{frame}

\begin{frame}
    \frametitle{Examples of Congruence}
    \begin{example}
        \begin{align}
            17 & \equiv 5 \pmod{12} \quad \text{because } 17 - 5 = 12 = 1 \cdot 12   \\
            -3 & \equiv 9 \pmod{12} \quad \text{because } -3 - 9 = -12 = -1 \cdot 12 \\
            25 & \equiv 1 \pmod{12} \quad \text{because } 25 - 1 = 24 = 2 \cdot 12
        \end{align}
    \end{example}

    \vspace{0.5cm}

    \begin{block}{Verification by Division}
        \begin{itemize}
            \item $17 = 1 \cdot 12 + 5$ (remainder 5)
            \item $5 = 0 \cdot 12 + 5$ (remainder 5)
            \item Since both have remainder 5, $17 \equiv 5 \pmod{12}$
        \end{itemize}
    \end{block}
\end{frame}

\section{Properties of Congruence}

\begin{frame}
    \frametitle{Properties of Congruence}
    \begin{theorem}[Basic Properties]
        Let $n$ be a positive integer. Then congruence modulo $n$ is:
        \begin{enumerate}
            \item \textbf{Reflexive:} $a \equiv a \pmod{n}$
            \item \textbf{Symmetric:} If $a \equiv b \pmod{n}$, then $b \equiv a \pmod{n}$
            \item \textbf{Transitive:} If $a \equiv b \pmod{n}$ and $b \equiv c \pmod{n}$, then $a \equiv c \pmod{n}$
        \end{enumerate}
    \end{theorem}

\end{frame}

\begin{frame}
    \frametitle{Arithmetic Properties}
    \begin{theorem}[Arithmetic with Congruences]
        If $a \equiv b \pmod{n}$ and $c \equiv d \pmod{n}$, then:
        \begin{enumerate}
            \item $a + c \equiv b + d \pmod{n}$
            \item $a - c \equiv b - d \pmod{n}$
            \item $ac \equiv bd \pmod{n}$
            \item $a^k \equiv b^k \pmod{n}$ for any positive integer $k$
        \end{enumerate}
    \end{theorem}

    \begin{example}
        Since $17 \equiv 5 \pmod{12}$ and $13 \equiv 1 \pmod{12}$:
        \begin{align}
            17 + 13     & \equiv 5 + 1 \equiv 6 \pmod{12}     \\
            17 \cdot 13 & \equiv 5 \cdot 1 \equiv 5 \pmod{12}
        \end{align}
    \end{example}
\end{frame}



\section{Applications}

\begin{frame}
    \frametitle{Computing with Large Numbers}
    \begin{example}
        Compute $2^{100} \pmod{7}$.
    \end{example}

    \begin{block}{Solution}
        First, let's find the pattern of powers of 2 modulo 7:
        \begin{align}
            2^1 & \equiv 2 \pmod{7}                                                                 \\
            2^2 & \equiv 4 \pmod{7}                                                                 \\
            2^3 & = 8 \equiv 1 \pmod{7} \quad \text{(loops back to 1)}                              \\
            2^4 & \equiv 2^3 \cdot 2^1 \equiv 1 \cdot 2 \equiv 2 \pmod{7} \quad \text{(back to 2!)}
        \end{align}

        The pattern repeats every 3 steps: $2, 4, 1, 2, 4, 1, \ldots$
    \end{block}
\end{frame}

\begin{frame}
    \frametitle{Computing with Large Numbers (continued)}
    \begin{block}{Solution (continued)}
        Since $100 = 3 \cdot 33 + 1$, we have:
        \begin{align}
            2^{100} & = 2^{3 \cdot 33 + 1}               \\
                    & = (2^3)^{33} \cdot 2^1             \\
                    & \equiv 1^{33} \cdot 2 \pmod{7}     \\
                    & \equiv 1 \cdot 2 \equiv 2 \pmod{7}
        \end{align}
    \end{block}
\end{frame}

\begin{frame}
    \frametitle{Finding Unit Digits}
    \begin{example}
        Find the unit digit of $23^{343}$.
    \end{example}

    \begin{block}{Solution}
        To find the unit digit, we compute $23^{343} \pmod{10}$.

        Since $23 \equiv 3 \pmod{10}$, we need to find $3^{343} \pmod{10}$.

        First, let's find the pattern of powers of 3 modulo 10:
        \begin{align}
            3^1 & \equiv 3 \pmod{10}                                                                           \\
            3^2 & \equiv 9 \pmod{10}                                                                           \\
            3^3 & = 27 \equiv 7 \pmod{10}                                                                      \\
            3^4 & = 3^3 \cdot 3^1 \equiv 7 \cdot 3 \equiv 21 \equiv 1 \pmod{10} \quad \text{(cycle complete!)}
        \end{align}

        The pattern repeats every 4 steps: $3, 9, 7, 1, 3, 9, 7, 1, \ldots$
    \end{block}
\end{frame}

\begin{frame}
    \frametitle{Finding Unit Digits (continued)}
    \begin{block}{Solution (continued)}
        Since $343 = 4 \cdot 85 + 3$, we have:
        \begin{align}
            3^{343} & = 3^{4 \cdot 85 + 3}                \\
                    & = (3^4)^{85} \cdot 3^3              \\
                    & \equiv 1^{85} \cdot 7 \pmod{10}     \\
                    & \equiv 1 \cdot 7 \equiv 7 \pmod{10}
        \end{align}

        Therefore, the unit digit of $23^{343}$ is $\boxed{7}$.
    \end{block}
\end{frame}

\begin{frame}
    \frametitle{Exercise}
    \begin{block}{Problem 27, POSN Computer 2562}
        Find the remainder of $2018^{2019} + 2019^{2020} + 2020^{2021}$ when divided by 13.
    \end{block}

\end{frame}

\begin{frame}
    \frametitle{Days of the Week}
    \begin{block}{Day Calculation}
        We can use modular arithmetic to determine what day of the week a given date falls on.

        Each day corresponds to a number modulo 7:
        \begin{itemize}
            \item Sunday = 0, Monday = 1, ..., Saturday = 6
        \end{itemize}
    \end{block}

    \begin{example}
        If today is Wednesday (day 3), what day will it be in 100 days?

        $3 + 100 = 103$

        $103 = 14 \cdot 7 + 5$, so $103 \equiv 5 \pmod{7}$

        Day 5 corresponds to Friday.
    \end{example}
\end{frame}

\section{GCD and LCM}

\begin{frame}
    \frametitle{Greatest Common Divisor (GCD)}
    \begin{definition}
        The \textbf{greatest common divisor} of two integers $a$ and $b$ (not both zero) is the largest positive integer that divides both $a$ and $b$.

        We write: $\gcd(a,b)$ or $(a,b)$
    \end{definition}

    \vspace{0.5cm}

    \begin{block}{Properties}
        \begin{itemize}
            \item $\gcd(a,0) = |a|$ for any non-zero integer $a$
            \item $\gcd(a,b) = \gcd(b,a)$ (symmetry)
            \item $\gcd(a,b) = \gcd(a, b \bmod a)$ if $a > 0$
        \end{itemize}
    \end{block}
\end{frame}

\begin{frame}
    \frametitle{Euclidean Algorithm}

    \begin{example}
        Find $\gcd(252, 105)$:
        \begin{align}
            gcd(252, 105) & = \gcd(105, 252 \mod 105) \\
                          & = \gcd(105, 42)           \\
                          & = \gcd(42, 105 \mod 42)   \\
                          & = \gcd(42, 21)            \\
                          & = \gcd(21, 42 \mod 21)    \\
                          & = \gcd(21, 0)             \\
                          & = 21
        \end{align}
        Therefore, $\gcd(252,105) = 21$.
    \end{example}
\end{frame}

\begin{frame}
    \frametitle{Bézout's Identity}
    \begin{theorem}[Bézout's Identity]
        For any integers $a$ and $b$ (not both zero), there exist integers $x$ and $y$ such that:
        $$ax + by = \gcd(a,b)$$

        These integers $x$ and $y$ are called \textbf{Bézout coefficients}.
    \end{theorem}

    \vspace{0.5cm}

    \begin{block}{Extended Euclidean Algorithm}
        We can find the Bézout coefficients by working backwards through the Euclidean algorithm.
    \end{block}
\end{frame}

\begin{frame}
    \frametitle{Bézout's Identity Example}
    \begin{example}
        Find integers $x$ and $y$ such that $252x + 105y = \gcd(252,105) = 21$.
    \end{example}

    \begin{block}{Solution}
        Working backwards from our Euclidean algorithm:
        \begin{align}
            21 & = 105 - 2 \cdot 42                  \\
               & = 105 - 2 \cdot (252 - 2 \cdot 105) \\
               & = 105 - 2 \cdot 252 + 4 \cdot 105   \\
               & = 5 \cdot 105 - 2 \cdot 252         \\
               & = (-2) \cdot 252 + 5 \cdot 105
        \end{align}

        Therefore, $x = -2$ and $y = 5$.

        \textbf{Verification:} $252(-2) + 105(5) = -504 + 525 = 21$ \checkmark
    \end{block}
\end{frame}

\begin{frame}
    \frametitle{Least Common Multiple (LCM)}
    \begin{definition}
        The \textbf{least common multiple} of two positive integers $a$ and $b$ is the smallest positive integer that is divisible by both $a$ and $b$.

        We write: $\text{lcm}(a,b)$ or $[a,b]$
    \end{definition}

    \vspace{0.5cm}

    \begin{theorem}[Fundamental Relationship]
        For any positive integers $a$ and $b$:
        $$\gcd(a,b) \times \text{lcm}(a,b) = a \times b$$
    \end{theorem}

\end{frame}

\begin{frame}
    \frametitle{GCD and LCM Example}
    \begin{example}
        If two number $a$ and $b$ have $\gcd(a,b) = 10$ and $\text{lcm}(a,b) = 100$, find $a \times b$.
    \end{example}

    \begin{block}{Solution}
        We know that $\gcd(a,b) \times \text{lcm}(a,b) = a \times b$.

        Therefore, $a \times b = 10 \times 100 = 1000$.
    \end{block}


\end{frame}

\end{document}

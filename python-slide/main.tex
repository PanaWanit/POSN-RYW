\documentclass[aspectratio=43]{beamer}
\usepackage{graphicx} % Required for inserting images
\usepackage{colortbl}
\usepackage{adjustbox}
\usepackage{mathtools}
\usepackage{hhline}

\usepackage{listings}
\usepackage{color}
\usepackage{natbib}
\usepackage{xcolor}

\definecolor{myblue}{RGB}{20,105,176}
\definecolor{constants}{RGB}{127,39,84}
\definecolor{character}{RGB}{100,169,57}
\setbeamertemplate{footline}[page number]
\lstdefinestyle{deltaj}{
        belowcaptionskip=1\baselineskip,
        breaklines=true,
        columns=fullflexible,
        frame=single,
        xleftmargin=\parindent,
        language=Python,
        escapeinside={(*}{*)},
        stepnumber=1,
        tabsize=2,
        numberblanklines=false,
        numberstyle=\normalfont\footnotesize,
        basicstyle=\footnotesize\ttfamily,
        keywordstyle=\bfseries\color{green!40!black},
        commentstyle=\itshape\color{purple!40!black},
        identifierstyle=\color{blue},
        stringstyle=\color{orange}, 
        showstringspaces=false
    }
\lstset{style=deltaj}

\beamertemplatenavigationsymbolsempty
\usetheme{Ilmenau}
% \usecolortheme{beaver}

\title[Basic Python]{Basic Python Programming}
\subtitle[POSN Computer]{for POSN Computer Qualification Exam}
\author[Pana W.]{Pana Wanitchollakit}
\date{2024}


\AtBeginSection[]{
\begin{frame}
  \vfill
  \centering
  \begin{beamercolorbox}[sep=8pt,center,shadow=true,rounded=true]{title}
    \usebeamerfont{title}\insertsectionhead\par%
  \end{beamercolorbox}
  \vfill
  \end{frame}
}

\begin{document}

\frame{\titlepage}

\begin{frame}{Outline}
    \only<1>{\tableofcontents[sections={1-3}]}
    \only<2>{\tableofcontents[sections={4-}]}
\end{frame}


\section{Basic}

\subsection[Data types]{Data types}
\begin{frame}{Data types}
    \begin{block}{Data types}
        \lstinputlisting{source/datatypes.py}
    \end{block}
    \begin{block}{Type conversion}
        \lstinputlisting{source/typeconversion.py}
    \end{block}
\end{frame}


\subsection[I/O]{Input \& Output}
\begin{frame}{Input}
    A function named \textbf{input} is used for only \textbf{string} input. \\
    The \textbf{input} function will receive input until you press the $<$Enter$>$.
    \begin{block}{Input}
        \lstinputlisting{source/input.py}
        If you want an integer input, you must convert it to an integer, i.e.,
        \lstinputlisting{source/int_input.py}
    \end{block}
\end{frame}

\begin{frame}{Output}
    A \textbf{print} function is used to show output on your commandline. \\
    Note: \textbf{print} will add an extra character, namely a newline character "\textbackslash n" at the end of your output.
    \begin{block}{Output}
        \lstinputlisting{source/print.py}
    \end{block}
\end{frame}

\subsection[Operator]{Operator}
\subsubsection{String operator}
\begin{frame}{String operator}
    \begin{block}{Plus operator (+)}
        We can concatenate the string using \textbf{+} operator
        \lstinputlisting{source/plus_string.py}
    \end{block}
    \begin{block}{Multiplication operator (*)}
        Just like multiplying in arithmetic, we concatenate (plus) a string multiple times instead.
        \lstinputlisting{source/mul_string.py}
    \end{block}
\end{frame}

\subsubsection{Arithmetic Operators}
\begin{frame}{Arithmetic Operators}
    \begin{block}{Arithmetic}
        \lstinputlisting{source/arithmetic.py}
    \end{block}
\end{frame}

\begin{frame}{Arithmetic Operators Precedence}
    \centering
    \begin{tabular}{|c|c|c|c|}
        \hline
        Order & Operator     & Associativity Type \\
        \hline
        1     & ()           & -                  \\
        \hline
        2     & **           & right to left      \\
        \hline
        3     & *, /, //, \% & left to right      \\
        \hline
        4     & +, -         & left to right      \\
        \hline
    \end{tabular}
\end{frame}

\begin{frame}
    \begin{example}
        \lstinputlisting{source/arith_precedence/e1.py}
        $$
            10 + (3\times4) + (2^3 \times 4) - \left((4\times5) - \frac{25}{5^2}\right) + 5 = 40
        $$
    \end{example}
    \begin{example}
        \lstinputlisting{source/arith_precedence/e2.py}
        $$
            4 \times 5^{3^2} \times 7 = 4 \times 5^9 \times 7
        $$
    \end{example}
\end{frame}

\begin{frame}
    \begin{example}
        \lstinputlisting{source/arith_precedence/e3.py}
        $$
            \left \lfloor \frac{\left \lfloor \frac{5 \times 3}{4} \right\rfloor \times 10}{7} \right \rfloor
        $$
        Note: \\
        Floor function \href{https://en.wikipedia.org/wiki/Floor_and_ceiling_functions}{\beamergotobutton{Definition}} : $\lfloor x \rfloor$ \\
        Example: $\lfloor -2.5 \rfloor = -3$, $\lfloor 2.5 \rfloor = 2$
    \end{example}
\end{frame}

\section[String and List]{String and List}
\subsection{List}
\begin{frame}{List}
    In short, it's a box of variables. \\
    \textbf{String} is a special type of list (all elements of a list are characters).
    \begin{example}
        \lstinputlisting{source/list.py}
    \end{example}

    \begin{block}{List methods (function)}
        \lstinputlisting{source/list_method.py}
    \end{block}
\end{frame}

\subsection{Operators}
\begin{frame}{String \& List Operators}
    \begin{block}{Length}
        The \textbf{len} function can be used to count the length of a string.
        \lstinputlisting{source/length.py}
    \end{block}
\end{frame}

\begin{frame}{}
    \begin{block}{Indexing}
        You can access a character in string or a element in list with a [ ]. \\
        Note: Indexing in Python starts with 0 and can be accessed with a negative index (reversed index).
        \[
            \begin{array}{|c|c|c|c|c|c|c|c|c|c|c|c|c|}
                \hline
                idx     & 0   & 1   & 2  & 3  & 4  & 5  & 6  & 7  & 8  & 9  & 10 \\
                \hline
                str     & H   & e   & l  & l  & o  & \  & W  & o  & r  & l  & d  \\
                \hline
                idx_{r} & -11 & -10 & -9 & -8 & -7 & -6 & -5 & -4 & -3 & -2 & -1 \\
                \hline
            \end{array}
        \]
        \lstinputlisting{source/string_indexing.py}

    \end{block}
\end{frame}

\section[Control flow]{Control flow statements }
\subsection[Boolean operator]{Boolean operator}

\begin{frame}{Boolean operators}
    \begin{block}{Relation operators}
        The relation operators yield \textit{boolean} values, i.e., \textbf{True} or \textbf{False}.
        \begin{center}
            \begin{tabular}{|c|c|c|c|c|c|c|}
                \hline
                Python & $<$ & $>$ & $<=$  & $>=$  & $!=$  & ==  \\
                \hline
                Math   & $<$ & $>$ & $\le$ & $\ge$ & $\ne$ & $=$ \\
                \hline
            \end{tabular}
        \end{center}
    \end{block}

    \begin{block}{Logical operators}
        Apparently, boolean is just a \textbf{proposition} in logic.
        \begin{center}
            \begin{tabular}{|c|c|c|c|}
                \hline
                Python      & \textbf{and} & \textbf{or} & \textbf{not}     \\
                \hline
                Logic(Math) & $\land$      & $\lor$      & $\neg$ or $\sim$ \\
                \hline
            \end{tabular}
        \end{center}
    \end{block}

\end{frame}

\begin{frame}
    \begin{block}{De Morgan's Law}
        \begin{itemize}
            \item $\neg(p \land q) \equiv \neg p \lor \neg q$
            \item $\neg(p \lor q) \equiv \neg p \land \neg q$
        \end{itemize}
    \end{block}
    \begin{block}{Negation of Relation operators}
        \begin{center}
            \begin{tabular}{|c|c|c|c|c|c|c|}
                \hline
                p       & $a<b$    & $a>b$    & $a \le b$ & $a\ge b$ & $a\ne b$ & $a=b$    \\
                \hline
                $\neg$p & $a\ge b$ & $a\le b$ & $a>b$     & $a<b$    & $a=b$    & $a\ne b$ \\
                \hline
            \end{tabular}
        \end{center}
    \end{block}
\end{frame}

\begin{frame}{Boolean operator precedence}
    \textbf{Note:} Every boolean operator has lower precedence than all arithmetic operators.
    \begin{center}
        \begin{tabular}{|c|c|}
            \hline
            Order & Operator          \\
            \hline
            1     & $==,!=,<=,>=,>,<$ \\
            \hline
            2     & \textbf{not}      \\
            \hline
            3     & \textbf{and}      \\
            \hline
            4     & \textbf{or}       \\
            \hline
        \end{tabular}
    \end{center}
\end{frame}

\begin{frame}
    \begin{example}
        \lstinputlisting{source/boolean_expression.py}
    \end{example}
\end{frame}

\subsection[Boolean expression]{Boolean expression}
\begin{frame}{if-elif-else statement}
    \begin{block}{if else-if else block}
        \begin{itemize}
            \item \textbf{if} start condition(proposition).
            \item \textbf{elif} another condition if \textbf{if} is rejected
            \item \textbf{elif} another condition if the above \textbf{elif} is rejected
            \item \textbf{elif} another condition if the above \textbf{elif} is rejected
            \item $\vdots$
            \item \textbf{else} if all the conditions above are rejected.
        \end{itemize}
        The \textbf{if-else} blocks can contain many (or none) \textbf{elif} block and not be nesscessary to have an \textbf{else}. \\
        \textbf{Note}: Each \textbf{if-else} block performs action \textit{once} or \textit{none} (no \textbf{else}).
    \end{block}
\end{frame}

\begin{frame}
    \begin{example}
        \lstinputlisting{source/if_else.py}
        \begin{itemize}
            \item \makebox[2.5cm][l]{if $x \le 10$} : do if
            \item \makebox[2.5cm][l]{if $10 < x \le 25$} : do elif 1
            \item \makebox[2.5cm][l]{if $25 < x \le 50$} : do elif 2
            \item \makebox[2.5cm][l]{if $x > 50$} : do else
        \end{itemize}
    \end{example}
\end{frame}

\begin{frame}
    \begin{alertblock}{Many if vs if-else}
        \lstinputlisting{source/many_if.py}
        \begin{itemize}
            \item \makebox[2.5cm][l]{if $x \le 10$} : f ef1 ef2
            \item \makebox[2.5cm][l]{if $10<x \le 25$} : ef1 ef2
            \item \makebox[2.5cm][l]{if $25<x \le 50$} : ef2
            \item \makebox[2.5cm][l]{if $x > 50$} : e
        \end{itemize}
    \end{alertblock}

\end{frame}

\section{Loop}
\subsection{For Loop}
\begin{frame}{For Loop}
    \begin{block}{\textbf{range} $\href{https://docs.python.org/3/library/stdtypes.html\#ranges}{\beamergotobutton{Documentation}}$}
        \textbf{range} is an iterable object (can be converted into a \textbf{list}).
    \end{block}
    \begin{example}
        \lstinputlisting{source/range.py}
        Note: \textbf{stop} exclusive.
    \end{example}
\end{frame}

\begin{frame}
    \begin{block}{\textbf{for} keyword}
        The \textbf{for} keyword is used to iterate an iterable object with the keyword \textbf{in}.
    \end{block}
    \begin{example}
        \lstinputlisting{source/for.py}
        \textbf{Note}: \textbf{end} keyword in \textbf{print} change "\textbackslash n" to the specified string.
    \end{example}
\end{frame}

\subsection{While Loop}
\begin{frame}{While Loop}
    \begin{block}{\textbf{while} keyword}
        The \textbf{while} keyword is like the \textbf{if-else} block, but \textbf{while} loop does action until the condition is rejected.
    \end{block}
    \begin{block}{Variable assignment}
        You can assign new value in declared variable.
        \lstinputlisting{source/assignment.py}
    \end{block}
\end{frame}

\begin{frame}
    \begin{example}
        \lstinputlisting{source/while.py}
    \end{example}
\end{frame}

\subsection[Nested Loop]{Nested Loop}
\begin{frame}{Nested Loop}
    Loop can be nested.
    \begin{columns}
        \begin{column}{0.5\textwidth}
            \begin{example}
                \lstinputlisting{source/nested.py}
            \end{example}
        \end{column}
        \begin{column}{0.5\textwidth}  %%<--- here
            \begin{block}{Output}
                * \\
                ** \\
                *** \\
                **** \\
                ***** \\
                ****** \\
                ******* \\
                ******** \\
                ********* \\
            \end{block}

        \end{column}
    \end{columns}
\end{frame}


\begin{frame}
    \begin{columns}
        \begin{column}{0.5\textwidth}
            \begin{example}
                \lstinputlisting{source/nested2.py}
            \end{example}
        \end{column}
        \begin{column}{0.5\textwidth}
            \begin{block}{Output}
                yxxxxxxxxx\\
                yyxxxxxxxx\\
                yyyxxxxxxx\\
                yyyyxxxxxx\\
                yyyyyxxxxx\\
                yyyyyyxxxx\\
                yyyyyyyxxx\\
                yyyyyyyyxx\\
                yyyyyyyyyx\\
                yyyyyyyyyy
            \end{block}

        \end{column}
    \end{columns}
\end{frame}

\end{document}